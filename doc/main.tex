\documentclass[a4paper,12pt]{article}
\usepackage[english]{babel}
\usepackage[utf8]{inputenc}
\usepackage{setspace}

% Larger borders -- we do not want do waste paper, even if it is only paper on screen =)
\usepackage[top=2.5cm, bottom=2.5cm, left=2cm, right=2cm]{geometry}
% Remove auto indentation of paragraphs.
\setlength\parindent{0pt}

% Palatino font (nicer serif font: Times is for oldies)
\renewcommand*\rmdefault{ppl}

% Nested itemize list bullet style
\renewcommand{\labelitemi}{$\bullet$}
\renewcommand{\labelitemii}{$\circ$}
\renewcommand{\labelitemiii}{--}

% Math packages
\usepackage{mathtools}
\usepackage{amsmath}
\usepackage{amsfonts}
\usepackage{amssymb}

% Graphic packages
\usepackage{graphicx}
\usepackage{float}
\usepackage{adjustbox}
\usepackage{tikz}
\usepackage{forest,array}
\usetikzlibrary{shadows}

% Graphs styles
\forestset{
  giombatree/.style={
    for tree={
      grow = east,
      parent anchor=east,
      child anchor=west,
      edge={rounded corners=2mm},
      fill=violet!5,
      drop shadow,
      l sep=10mm,
      edge path={
        \noexpand\path [draw, \forestoption{edge}] (!u.parent anchor) -- +(5mm,0) -- (.child anchor)\forestoption{edge label};
      }
    }
  }
}
\forestset{
  qtree/.style={
    for tree={
      parent anchor=south,
      child anchor=north,
      align=center,
      edge={rounded corners=2mm},
      fill=violet!5,
      drop shadow,
      l sep=10mm,
    }
  }
}

% Project Name
\newcommand{\projectname}{NetPP}

% Hides ugly links from the index
\usepackage[hidelinks]{hyperref}
% Landscape format pdf pagess
\usepackage{pdflscape}

\begin{document}
\pagenumbering{gobble}

{\setstretch{1.0}
  \begin{titlepage}
  	\centering
  	\includegraphics[width=6cm]{img/unipi.pdf}\par
    \vspace{1.5cm}
    {\Large Department of Information Engineering \par}
  	\vspace{1.5cm}
  	{\huge\textsc{\projectname{}}\par}
    \vspace{0.5cm}
    {\Large Advanced Networks and Wireless Systems project \par}
  	\vspace{2cm}
  	Francesco \textsc{Barbarulo}\par
  	Bruk Tekalgne \textsc{Gurmesa}\par
    Giovan Battista \textsc{Rolandi}

  	\vfill

    % Bottom of the page
  	{\large A.Y. 2019-2020\par}
  \end{titlepage}
}

\tableofcontents  % do we really need it? it should be a very quick small doc TODO

\pagenumbering{arabic}

\section{Introduction}
\projectname{} is a simulation of a sensors network which makes use of the following stack:
\begin{itemize}
  \item wireless IEEE 802.15.4
  \item 6LowPAN
  \item IPv6 (routing via RPL)
  \item UDP
  \item CoAP
\end{itemize}

It is developed on Z1 nodes with Contiki OS, and simulated inside Cooja.
A Java CoAP cache proxy has been developed using Eclipse Californium library, and all CoAP messages are RFC8428 (SenML) compliant.
The Trickle algorithm used inside RPL has been properly tuned in order to deal with a maximum of 40 wireless nodes randomly and uniformly distributed, and a maximum of 4 hops. % TODO please make sure it really is like this

\section{Overview}
Usage of RPL objective function 0 is mandatory by specification.
It is designed to find the nearest grounded (ie. connected to external network) root \cite{rfc:rplof0}.

\section{Trickle Parameters}
Depending on the specific network we want to develop, three different goals can be of interest:
\begin{itemize}
  \item energy consumption
  \item network formation time
  \item network stretch
\end{itemize}

Depending on the most relevant goal for the specific application, the Trickle algorithm used in RPL can be tuned via two parameters:
\begin{itemize}
  \item $k$
  \item $I_{min}$
\end{itemize}

Since we decided that \$somegoal was more important, and we read mingozzi, %% TODO
from \cite{bib:mingozzi:tricklef} we can infer that... % TODO infer something please

%% ETX expected transmissions %% TODO should we put this somewhere?

\begin{thebibliography}{9}

\bibitem{bib:mingozzi:tricklef}
  Mingozzi Enzo, Vallati Carlo --
  \emph{Trickle-F Fair broadcast suppression to improve energy-efficient route formation with the RPL routing protocol} --
  Università di Pisa
  (2013)

\bibitem{rfc:rplof0}
  Pascal Thubert --
  \emph{RFC6552 (Proposed Standard) Objective Function Zero for the Routing Protocol for Low-Power and Lossy Networks (RPL)} --
  Cisco Systems
  (2012)


\end{thebibliography}

\end{document}
